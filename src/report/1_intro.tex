The intersection of computational techniques with humanities research has historically encountered both skepticism and gradual acceptance \parencite{macroanalysis_digital_humanities, big_data_history}. As early as the 1980s, the legitimacy of computer-based work in humanities was under debate, often seen as too ``mechanical'' for the exploration of human culture and history \parencite{olsen_1993, hockey_2004}. It wasn't until the turn of the century when advanced processing capabilities, simpler programming interfaces, and greater data availability came into fruition that the potential of ``digital humanities'' as a field of academia began to be recognised more broadly \parencite{dh_history}.

This project is uniquely positioned at such an intersection, specifically intertwining humanities, anatomy, and data science. The broad aim of our project is to attain a better understanding of changes in anatomy body procurement practices during the formation of modern Australia. Central to this are the Anatomy Registers from The University of Sydney School of Medical Science, which chronicle over a century of body acquisition for anatomical study. This extensive dataset not only shows the evolution of medical and scientific practices in the nineteenth and twentieth centuries, but also opens a window to the sociocultural and ethical dimensions of the era. By exploring these registers, we are exploring how societal values, medical ethics, and the law all interact with the scientific pursuit of knowledge.

\subsection{Background}

In the digital era, it might seem that all necessary data for addressing contemporary challenges is readily available in a structured, digital format. In reality, researchers are repeatedly falling back onto the troves of information locked away in historical records to make ground on their inquiries -- information which often remains underutilised due to its non-digital or unstructured format \parencite{big_data_history}.

\paragraph{Sea level records}{One domain which has found good use of digitising archival records is the domain of geological sciences, specifically, the field of oceanography. A rise in global sea levels is expected to be one of the most devastating and costly effects of climate change in the next century \parencite{mimura_2013_sea}, and thus, having accurate data on sea levels across the world is essential for scientists to correctly understand the mechanisms involved behind these trends \parencite{marcos_2011_cadiz}. However, historical sea level data -- crucial for long-term analyses -- often only exists in analogue form, as digital recording only became prominent in the late 20th century. Consequently, current datasets might be inadequate for centennial-scale forecasting \parencite{talke_2020_columbia_river}. Recognising this gap, extensive digitisation efforts have emerged globally, including in Spain \parencite{marcos_2011_cadiz, marcos_2013_tenerife, marcos_2021_spain}, North America \parencite{talke_2014_new_york, talke_2018_boston, talke_2020_columbia_river}, France \parencite{woppelmann_2014_marseille}, and the United Kingdom \parencite{inayatillah_2022_thames}. These projects have significantly contributed to databases such as \textit{Système d'Observation du Niveau des Eaux Littorales} (\href{https://www.sonel.org/?lang=en}{SONEL}), which support ongoing oceanographic research \parencite{lyszkowicz_2019_baltic}.}

\paragraph{Ship logbooks}{Researchers including Kostas Petrakis have aimed to uncover truths about the transition from sail to steam navigation in the Mediterranean through digitising and analysing ship logbooks from the nineteenth and twentieth centuries. Despite the challenges posed by outdated navigational references and archaic language unfamiliar to contemporary scholars, the records have helped trace out the expansion of maritime routes due to the advent of steam ships, and also the psychological impact on those experiencing these shifts firsthand: \textit{``In the margin of the logbook [the captain] wrote twice about his disappointment and frustration...`The screw broke and I don’t know what to do, one difficulty after the other. God help me because I am going to lose my mind!'''} \parencite{petrakis_2021_ship}. It would be rare for such emotional insights to be captured through traditional research involving shipping records or calendars.}

The above examples highlight the value of the research described in this project: transcribing and analysing historical data helps researchers attain a greater understanding of the physical and sociocultural changes occurring throughout history.

\subsection{Project Scope}

Originally the project aspirations were quite broad. For example, one of the initial aims of the project was to map advances in medical knowledge and diagnosis granularity through analysing changes in cause of death terminology. This was to be conducted through a technique called ``dynamic topic modelling'' \parencite{blei_2006_dtm}, which utilises unsupervised machine learning techniques to extract the $n$ most prevalent topics from a corpus of text. However, the limitations of using hand-transcribed records soon became apparent. For example:

\begin{enumerate}
    \item No formal data cleaning procedure had been conducted on the dataset;
    \item Historical records are usually transcribed with the aim of extracting an accurate account of events or people and not necessarily for the purposes of analysis. Due to this, no \textit{deliberate} alterations were made from the source material during transcription e.g. fixing misspellings;
    \item The process of digitising records written in traditional cursive scripts is incredibly laborious and error-prone.
\end{enumerate}

Furthermore, though analysis on the dataset had been conducted before, none focused on geographical location or cause of death and as such these attributes were still in their raw form. The project aim thus shifted to (1) using programming techniques to transform the transcriptions into useable data; and (2) completing a geographical mapping of the place of death of all bodies received.