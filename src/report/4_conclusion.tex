\section{Conclusion}

The exploration of the Anatomy Registers from The University of Sydney Medical School has proven both challenging and enlightening. This project illuminated changes in historical practices and societal attitudes towards body donation, while reinforcing the value of digitising and analyzing historical datasets.

Working with manually transcribed data introduces significant complexities, primarily due to the extensive data cleaning and validation required to ensure analysis reliability. Transforming the "place of death" data into a structured format enabled us to identify geographical patterns and temporal changes in body donation sources. Our geospatial analyses, particularly the animated choropleth map, visually narrated the evolution of societal norms from a reliance on unclaimed bodies from public institutions to an increase in donations from private residences, highlighting a significant shift in public perception towards body donation.

Nevertheless, the project encountered some limitations, especially in data categorisation. The large number of entries labeled as "Other" highlights the potential for enhancements in methodology through employing advanced technologies like machine learning and natural language processing.

Future endeavors will likely focus on further refining data standardization processes and exploring our initial research questions, such as using the cause of death attribute to trace the emergence of new medical fields like psychiatry through dynamic topic modeling. This could potentially reveal historical trends in medical terminology and diagnostics.

In conclusion, while this project has provided valuable insights and advanced the field of digital humanities in relation to anatomical history, it also underscores the continuous need for innovation and reflection in the methodologies we employ to examine our past so that we may learn from it.